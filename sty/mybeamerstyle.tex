\documentclass[utf8]{beamer}
\usepackage{etex}
%%%%%%%%%%%%%%%%%%%%%%%%%%%%%%%%%%%%%%%%%%%%%%%%%%%%%%%%%%%%%%%%%%%%%%%%%%
% There are a number of symbols (e.g., \Square) that are defined by      %
% multiple packages.  In order to typeset all the variants in this       %
% document, we have to give glyph a unique name.  To do that, we define  %
% \savesymbol{XXX}, which renames a symbol from \XXX to \origXXX, and    %
% \restoresymbols{yyy}{XXX}, which renames \origXXX back to \XXX and     %
% defines a new command, \yyyXXX, which corresponds to the most recently %
% loaded version of \XXX.                                                %
%                                                                        %

% Save a symbol that we know is going to get redefined.
\def\savesymbol#1{%
  \expandafter\let\expandafter\origsym\expandafter=\csname#1\endcsname
  \expandafter\let\csname orig#1\endcsname=\origsym
  \expandafter\let\csname#1\endcsname=\relax
}

% Restore a previously saved symbol, and rename the current one.
\def\restoresymbol#1#2{%
  \expandafter\let\expandafter\newsym\expandafter=\csname#2\endcsname
  \expandafter\global\expandafter\let\csname#1#2\endcsname=\newsym
  \expandafter\let\expandafter\origsym\expandafter=\csname orig#2\endcsname
  \expandafter\global\expandafter\let\csname#2\endcsname=\origsym
}

% Rather than create a rat's nest of \if statements, we keep the table
% whole and have each symbol conditionally appear.
\makeatletter
\newcommand{\trysym}[1]{\@ifundefined{#1}{\mbox{\tiny N/A}}{\csname#1\endcsname}}
\makeatother
%                                                                        %
%%%%%%%%%%%%%%%%%%%%%%%%%%%%%%%%%%%%%%%%%%%%%%%%%%%%%%%%%%%%%%%%%%%%%%%%%%

%---- usepackages ----
\usepackage{amscd}
\usepackage{amsmath,amssymb,amsfonts}%,mathrsfs}

%\usepackage{CJKutf8}
%\usepackage{CJKpunct,CJKspace}
%\punctstyle{CCT}
\usepackage{ctex}
\usepackage{xcolor}
\usepackage{pgf,tikz}
\usepackage{graphicx}
\usepackage{subfigure}
%\usepackage{picins}  % 图片嵌入段落宏包 比如照片
%\usepackage{pstool} % for matlabfrag.m figures
%\usepackage[english]{babel}
%\usepackage[utf8]{inputenc}
\usepackage{times}
%\usepackage[T1]{fontenc}
%\usepackage{booktabs}
% Or whatever. Note that the encoding and the font should match. If T1
% does not look nice, try deleting the line with the fontenc.
\usepackage{hyperref}
	\hypersetup{
		CJKbookmarks=true
		pdfpagemode={FullScreen}
	}
%---- set beamer page size to fit 16:9 or 16:10 screen ----
%---- uncomment next line for 16:9 aspect ratio ---
%\usepackage[orientation=landscape,size=custom,width=17.067,height=9.6,scale=0.5,debug]{beamerposter}
%---- uncomment next line for 16:10 aspect ratio ---
%\usepackage[orientation=landscape,size=custom,width=15.36,height=9.6,scale=0.5,debug]{beamerposter}

%---- 重定义字体、字号命令 ----
%	\newcommand{\songti}{\CJKfamily{song}}     % 宋体   (Windows自带simsun.ttf)
%	\newcommand{\kaishu}{\CJKfamily{kai}}      % 楷体   (Windows自带simkai.ttf)
	\newcommand{\lishu}{\CJKfamily{li}}
	\newcommand{\bodmtblk}{\CJKfamily{bodonimtblk}}
	% ---- 字号设置 ----
	\newcommand{\chuhao}{\fontsize{42pt}{\baselineskip}\selectfont}
	\newcommand{\xiaochuhao}{\fontsize{36pt}{\baselineskip}\selectfont}
	\newcommand{\yichu}{\fontsize{32pt}{\baselineskip}\selectfont}
	\newcommand{\yihao}{\fontsize{28pt}{\baselineskip}\selectfont}
	\newcommand{\erhao}{\fontsize{21pt}{\baselineskip}\selectfont}
	\newcommand{\xiaoerhao}{\fontsize{18pt}{\baselineskip}\selectfont}
	\newcommand{\sanhao}{\fontsize{15.75pt}{\baselineskip}\selectfont}
	\newcommand{\sihao}{\fontsize{14pt}{\baselineskip}\selectfont}
	\newcommand{\xiaosihao}{\fontsize{12pt}{\baselineskip}\selectfont}
	\newcommand{\wuhao}{\fontsize{10.5pt}{\baselineskip}\selectfont}
	\newcommand{\xiaowuhao}{\fontsize{9pt}{\baselineskip}\selectfont}
	\newcommand{\liuhao}{\fontsize{7.875pt}{\baselineskip}\selectfont}
	\newcommand{\qihao}{\fontsize{5.25pt}{\baselineskip}\selectfont}

\mode<presentation>
{
	%\usetheme{Warsaw}
	\usecolortheme{whale} % beaver, crane, wolverine - warm color theme
						% dove - black and white color theme
						% seahorse, shark - nice color theme
						% whale, dolphin - blue color theme, use 'outercolortheme' option for 'chamfered' inner theme
						% thu, EmoryBlue
  \setbeamercolor*{palette primary}{use=structure,bg=white}
	\usefonttheme{serif} % serif, structurebold, structuresmallcapsserif, professionalfonts
	\useinnertheme[realshadow]{chamfered}
	%\useinnertheme[realshadow, outercolortheme]{chamfered}
	\useoutertheme[nofootline]{wuerzburg}
	%\useoutertheme[nofootline, shadeframetitlebg]{wuerzburg}% options: glossy, nofootline, shadeframetitlebg
	\setbeamercovered{dynamic}
}

%---- setbeamertemplate ----
	%\setbeamertemplate[infolines theme]{headline}{}
	%\setbeamertemplate{footline}{} % ---- 删除底部信息条 ----
	% --删除底部导航工具栏, 替换为页数 或 logo--
	\usenavigationsymbolstemplate{\vbox{\hfill\structure{
		%\usebeamercolor[bg]{palette secondary}{\normalsize\bodmtblk\bfseries SUNIST}
		\includegraphics[height=0.3cm]{logo/sunist.pdf}
		%\includegraphics[height=0.3cm]{logo/sunist-wolverine.pdf}
		%\includegraphics[height=0.3cm]{logo/sunist-thu.pdf}
		%\includegraphics[height=0.3cm]{logo/sunist-whale.pdf}
		%\tiny\insertframenumber{} / \inserttotalframenumber
	}\hspace{1ex}\vspace*{1ex}}}
%---- 设置背景渐变色 ----
%	\setbeamertemplate{background canvas}[vertical shading][bottom=structure.fg!15,top=white]
	%\setbeamertemplate{frametitle}[horizontal shading][left=blue, right=black]
%---- pictures dir ----
\graphicspath{{pics/}}

%---- sunist logo ----
	% \pgfdeclareimage[height=0.5cm]{thu}{thu.JPG}
	% \logo{\pgfuseimage{thu}}
	%or
	%\logo{\includegraphics[height=0.3cm]{sunistlogo2.pdf}}

% Delete this, if you do not want the table of contents to pop up at
% the beginning of each subsection:
\AtBeginSubsection[]
{
	\begin{frame}<beamer>
		\frametitle{Outline}
		\tableofcontents[currentsection,currentsubsection]
	\end{frame}
}

% If you wish to uncover everything in a step-wise fashion, uncomment
% the following command:
%\beamerdefaultoverlayspecification{<+->}

%---- includeonlyframes ----
%\includeonlyframes{inclonlyframelabel}

%---- set paragraph skip ----%
\setlength{\parskip}{1ex  plus  0.5ex  minus  0.2ex}

% ---- Change the bullets ----
\useitemizeitemtemplate{%
    \tiny\raise1.5pt\hbox{{$\blacksquare$}}%
}
\usesubitemizeitemtemplate{%
    \tiny\raise1.5pt\hbox{{$\blacklozenge$}}%
}
\usesubsubitemizeitemtemplate{%
    \tiny\raise1.5pt\hbox{{$\blacktriangleright$}}%
}
%\usesubsubsubitemizeitemtemplate{%
%    \tiny\raise1.5pt\hbox{{$\bullet$}}%
%}
% Plot a tiny plasma current graph
% Input:
%   #1 Plot data
%   #2 Mean current
\newcommand{\dataplot}[2]{%
    \begin{tikzpicture}[xscale=0.5, yscale=0.03]
        \begin{scope}[ycomb, yscale=0.5]
        	\draw[densely dotted, thin, black!50] (35,0)--(40,0);
            \draw[densely dotted, thin, black!50] (35,#2)--(40,#2);
            \draw[ultra thin, black!80] plot[smooth] #1;
            \node[above left, scale=0.35] at (35,0) {$0$};
            \node[below left, scale=0.35] at (35,#2) {$#2$};
        \end{scope}
    \end{tikzpicture}%
}

%\pgfdeclarehorizontalshading[frametitle.bg]{beamer@frametitleshade}{1em}{%
%	color(0pt)=(frametitle.bg);
%	color(\paperwidth)=(white)
%}

\newcommand{\thankyoupage}[1]{%
	\begin{frame}
		\begin{centering}
		\begin{beamercolorbox}[sep=3pt,center]{title}
			\usebeamerfont{title}\inserttitle\par%
			\ifx\insertsubtitle\@empty%
			\else%
				\vskip0.25em%
				{\usebeamerfont{subtitle}\usebeamercolor[fg]{subtitle}\insertsubtitle\par}%
			\fi%
		\end{beamercolorbox}%
		\end{centering}
		\vskip2em\par
		\begin{centering}
		\hfill
		\begin{beamercolorbox}[wd=0.8\linewidth,center,rounded=true]{palette secondary}
			\usebeamerfont{title  in  head/foot}
			\huge{#1}
		\end{beamercolorbox}
		\hfill
		\end{centering}
	\end{frame}
}

\newcommand<>{\hover}[3]{\uncover#4{%
	\begin{tikzpicture}[remember picture,overlay]%
		\draw[fill,opacity=0.4] (current page.south west)
			rectangle (current page.north east);
		\node at (current page.center) {%
			\begin{minipage}{{#1}}
			\begin{block}{{#2}}
				{#3}
			\end{block}
			\end{minipage}
		};
	\end{tikzpicture}}
}

\newcommand{\myfootnote}[1]{\footnote{\tiny{{#1}}}}

\newcommand{\backupframebegin}{
     \newcounter{framenumberbeforeappendix}
     \setcounter{framenumberbeforeappendix}{\value{framenumber}}
     \setbeamertemplate{headline}[wuerzburg theme appendix]
     \setbeamertemplate{footline}[wuerzburg theme appendix]
}
\newcommand{\backupframeend}{
   \addtocounter{framenumberbeforeappendix}{-\value{framenumber}}
   \addtocounter{framenumber}{\value{framenumberbeforeappendix}}
   \setbeamertemplate{headline}[wuerzburg theme]
   \setbeamertemplate{footline}[wuerzburg theme]
}
